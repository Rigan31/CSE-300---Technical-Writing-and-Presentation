\documentclass[18pt, a4paper]{article} %14 pt indicates the font size of the prepared document
\usepackage[utf8]{inputenc} %indicates the encoding of the document
\usepackage{color} %this package enables the use of colors.
\usepackage{graphicx}
\usepackage{multicol}
\usepackage{multirow}
\usepackage{enumitem}
\usepackage{amsmath}
\usepackage{pgf,tikz}

\title{CSE 300: Online Assignment}
\author{Md Shamsuzzoha Bayzid, Mahjabin Nahar, Md Shariful Islam Bhuyan,\\
and Md Saidur Rahman}
\date{June 2021}

\begin{document}
    \maketitle
    \section{Introduction}
    This assignment has been designed to assess the preparation of the students
    in writing scientific articles using \LaTeX. This assignment covers a variety of
    components that are commonly used in scientific manuscripts.
    \subsection{Equations}
    Let $C$ be a simple piecewise smooth curve that bounds a region $D$ in the\\
    plane. If $P(x, y)$ and $Q(x, y)$ have continuous partials in an open region\\
    containing $D$, then
    
    $$
       \int_C{}^{} P \,dx  + Q \,dy = \iint_D{}^{} \frac{\partial Q}{\partial x} - \frac{\partial P}{\partial y} \,dA 
    $$
    \par If $F$ is a vector field with third component $0$ defined on a domain $D$
enclosed by boundary $C$ then
    $$
        \oint_C{}^{} \textbf{F}. \,dr = \iint_D{}^{} (\nabla \times \textbf{F.k} \,dA
    $$
    \par Similarly, if $C$ is defined by $\textbf{r}(t) = <hx(t), y(t)>$
     $$
        \oint_C{}^{} \textbf{F.n} \,dr = \iint_D{}^{} (\nabla .\textbf{F} \,dA
    $$
    
    \subsection{Tables}
    We wish to place the Table at the bottom of the page.
    \subsection{Figures}
    We intend to put Figure \ref{fig: image} at the top of a page.
    
    \pagebreak
    \section{Conclusions}
    The major objectives of this assignment are listed below (please do not ignore
    the font sizes).
    \begin{itemize}
        \item \huge {To assess the ability of the students in preparing
            manuscripts in} \LaTeX.
        \item \Large {To see if the students have adequately practiced different
            aspects of writing in} \LaTeX.
        \item \large{To see if the students can add various basic components (e.g., tables,
            figures, equations) to a \LaTeX manuscript}
        \item To see if the students can leverage the available materials (both offline and
            online) to do something which has not explicitly been taught in the class
    \end{itemize}
    
    \pagebreak
    
    \begin{table}[h]
        \centering
        \begin{tabular}{|l||l|l|l|}
     \hline
     \multicolumn{4}{|c|}{Item List} \\
     \hline
     Item  Name  or  & ALPHA 2 Code & ALPHA 3 Code & Numeric Code \\
     Product Name & & & \\
     \hline
     \multirow{2}{*}{Item001} & \multirow{2}{*}{AF} & \multirow{2}{*}{AFG} & 001 \\
      &  &  & 002 \\
      \hline
     Item002 & AX & ALA & 003 \\
     \hline
     \multirow{4}{*}{Item003} & \multirow{4}{*}{AL} & \multirow{4}{*}{ALB} & 004 \\
      &  &  & 005 \\
      &  &  & 006 \\
      &  &  & 008 \\
      \hline
     \multirow{2}{*}{Item004} & \multirow{2}{*}{DZ} & \multirow{2}{*}{DZA} & 009 \\
      &  &  & 010 \\
      \hline
     \multirow{2}{*}{Item005} & \multirow{2}{*}{AS} & \multirow{2}{*}{ASM} & 011 \\
      &  &  & 012 \\
      \hline
     Item006 & AD & AND & 013 \\
     \hline
     Item007 & AO & AGO & 014 \\
     \hline
     \hline

\end{tabular}
        \caption{Caption}
        \label{tab: table}
    \end{table}
    
    \pagebreak
    
    \begin{figure}[t]
		\centering
		\scalebox{1}{-1} {\includegraphics[scale=0.3]{Figure3.pdf}}
		\includegraphics[scale=0.3]{figure3.pdf}
		
		\caption{\textbf{Same figure upside down}}
		\label{fig: image}
	\end{figure}

\end{document}
